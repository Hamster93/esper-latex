% !TEX encoding = UTF-8 Unicode
\chapter{Fazit}
Mit Esper lassen sich auch komplexe Event-Ströme Analysieren und Auswerten. Durch den In-Memory Ansatz, kann dies sehr effizient mit geringen Latenzen realisiert werden,
da Daten nicht auf der Festplatte bzw. in einer Datenbank gespeichert und wieder ausgelesen werden müssen.
\absatz
Konzepte wie Data-Windows (siehe \ref{Data-Windows}) erlauben es, eine Kausalität zwischen Events herzustellen.
Damit ist gemeint, dass nicht nur das aktuelle Event betrachtet, sondern auch die Events, welche zum aktuellen Event geführt haben.
\absatz
In der Umsetzung (siehe \ref{EventsVerarbeiten}) wird aber auch gezeigt, wie komplex die Weiterverarbeitung der Events werden kann. Die eingesetzten Statements und Konzepte wie Data-Windows (siehe \ref{Data-Windows}) spielen dort eine große Rolle.
\absatz
Ziel dieser Ausarbeitung war es, eine Einführung in Esper zu geben und die Grundlegenden Konzepte zu erläutern. Dieses Ziel wurde nach Meinung der Autoren erreicht, in der Arbeit werden die grundlegenden Konzepte von Esper erläutert und eine Anleitung für den Einsatz beschrieben.