% !TEX encoding = UTF-8 Unicode
\chapter{Motivation}
In Zeiten der Digitalisierung und BigData, sind Informationen in Echtzeit immer wichtiger. Riesige stetig steigende Datenmengen fordern die IT heraus, Lösungen zu finden, welche den Datenfluss handhaben können. Aufkommende evtl. unstrukturierte Daten sollen analysiert und anschließend zu aussagekräftigen Informationen aggregiert werden.
Hinzu kommt die Notwendigkeit, auf gewonnene Informationen reagieren zu können, wozu leichtgewichtige agile Geschäftsprozesse benötigt werden.
Denn durch rasch und stetig wandelnde Marktkonstellationen- und Bedingungen wird es immer wichtiger für Organisationen, sich schnell und individuell anpassen zu können. 
\absatz
Um den Überblick zu behalten benötigt es eine einheitliche, lose gekoppelten Architektur, welche in Echtzeit große Datenmengen, im Optimalfall auch in einem verteilten System, verarbeiten kann. Dies stellt auch die Softwareentwicklung vor Herausforderungen.
Erforderlich sind Tools und Bibliotheken, mit welchen die Ereignisverarbeitung skalierbar in Echtzeit durchgeführt werden kann. Tools wie Esper stellen Lösungen dar, um diesen Problemen durch \ac{CEP} Herr zu werden. Eine flexible Architektur und ein agiles Vorgehen wird mit der Ereignisorientierung möglich.
\absatz
Um den Studierenden an der HTWG zu ermöglichen, im Bereich Data Analytics, Fallstudien durchzuführen und Architekturen zu untersuchen, wird das Tool Esper als Repräsentant der ereignis-verarbeitenden Lösungen verwendet.
Das Ziel ist es, Grundlagen der Bibliothek zu vermitteln. Dies führt den Leser entlang eines Szenarios, von der Grundlagentheorie, über die Installation und erste Schritte hinweg, bis hin zum lauffähigen Beispiel.
Dabei richtet sich diese Arbeit an die Leser, welche eine eigene Lösung im Bereich Ereignisorientierung erstellen möchten.
Nach dem Einblick in Esper, sollte er in der Lage sein, die Lösung theoretisch zu verstehen und technisch umsetzen zu können.