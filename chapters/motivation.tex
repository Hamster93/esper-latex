% !TEX encoding = UTF-8 Unicode
\chapter{Motivation}
In Zeiten der Digitalisierung und BigData, sind Informationen in Echtzeit immer wichtiger.
Riesige, stetig steigende Datenmengen fordern die IT heraus, um Lösungen zu finden.
 
Hinzu kommt die Fähigkeit agil auf die Informationen reagieren zu können, wozu Agile und Geschäfts-Prozesse benötigt werden. 
Durch rasch und stetig wandelnde Marktkonstellationen- und Bedingungen wird es immer wichtiger für Organisationen, sich schnell und individuell anpassen zu können. Um den Überblick zu behalten benötigt es eine einheitliche, lose gekoppelten Architektur, welche in Echtzeit große Datenmengen, im Optimalfall auch in einem verteilten System, verarbeiten kann. Dies stellt auch die Softwareentwicklung vor Herausforderungen.
Erforderlich sind Tools und Bibliotheken, mit welchen die Ereignisverarbeitung skalierbar in Echtzeit durchgeführt werden kann. Tools wie Esper stellen Lösungen dar, um diesen Problemen durch CEP mithilfe einer Event-Engine entgegenzuwirken. Hierdurch soll eine flexible Architektur und resultierend hieraus ein agiles Vorgehen ermöglicht werden.

Um den Studierenden an der HTWG zu ermöglichen, im Bereich Data Analytics, Fallstudien durchzuführen und Architekturen zu untersuchen, wird das Tool Esper, als Repräsentant der Event-verarbeitenden Lösungen, eingeführt.
Das Ziel ist die Grundlagen von Esper zu vermitteln. Dies führt den Leser von der Grundlagentheorie, über die Installation und erste Schritte hinweg, bis hin zum lauffähigen Beispiel.
Dabei richtet sich diese Arbeit an die Leser, welche eine eigene Lösung im Bereich Ereignisorientierung erstellen möchten.
Dem Leser soll einen Einblick in Esper ermöglicht werden, um danach in der Lage zu sein, die Lösung zu verstehen und technisch umsetzen zu können.